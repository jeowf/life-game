Conway\textquotesingle{}s Life Game

Desenvolvido por Alison Hedigliranes da Silva e Felipe Morais da Silva.

\section*{Objetivos}


\begin{DoxyItemize}
\item \mbox{[} \mbox{]} Modelar classes
\item \mbox{[} \mbox{]} Organizar código em sub-\/pastas
\item \mbox{[} \mbox{]} Criar makefile
\item \mbox{[} \mbox{]} Implementar classes
\item \mbox{[} \mbox{]} Implementar programa
\item \mbox{[} \mbox{]} Documentar código
\end{DoxyItemize}

\subsection*{Descrição}

O jogo consiste em um simulador de vida em que dado uma geração inicial de células, baseado num conjunto de regras, algumas dessas morrem e outras nascem. As regras usam o conceito de vizinhança que são as oito células que tocam outra verticalmente, horizontalmente ou diagonalmente. As regras são\+:
\begin{DoxyItemize}
\item Regra 1\+: Se uma célula está viva, mas o número de vizinhos vivos é menor ou igual a um, na próxima geração a célula morrerá de solidão.
\item Regra 2\+: Se uma célula está viva e tem quatro ou mais vizinhos vivos, na próxima geração a célula morrerá sufocada devido a superpopulação.
\item Regra 3\+: Uma célula viva com dois ou três vizinhos vivos, na próxima geração permanecerá viva.
\item Regra 4\+: Se uma célula está morta, então na próxima geração ela se tornará viva se possuir exatamente três vizinhos vivos. Se possuir uma quantidade de vizinhos vivos diferente de três, a célula permanecerá morta.
\item Regra 5\+: Todos os nascimentos e mortes acontecem exatamente ao mesmo tempo, portanto células que estão morrendo podem ajudar outras a nascer, mas não podem prevenir a morte de outros pela redução da superpopulação; da mesma forma, células que estão nascendo não irão preservar ou matar células vivas na geração anterior.
\end{DoxyItemize}

\subsection*{Execução}

Para gerar o executavel do programa rode dentro da pasta\+:


\begin{DoxyCode}
1 Make
\end{DoxyCode}
 Assim será gerado um executável nomeado \char`\"{}life\char`\"{}.

Para executar o programa use\+:


\begin{DoxyCode}
1 ./life <input\_cfg\_file> (<output\_cfg\_evolution\_file>)
\end{DoxyCode}


Onde o primeiro parâmetro é um arquivo de entrada contendo o tamanho da matriz, caractere desejado para indicar células vivas e o estado inicial das células. O segundo parâmetro é opcional onde pode ser passado o nome de uma arquivo de saida que guardará o histórico de gerações.

Para apagar o executável\+:


\begin{DoxyCode}
1 make clean
\end{DoxyCode}


\subsection*{Tratamento de Erros}

Atualmente o programa trata os seguintes erros\+:


\begin{DoxyItemize}
\item Ausência ou excesso de argumentos na linha de comando;
\item Erros na leitura do arquivo de entrada.
\end{DoxyItemize}

\subsection*{Arquivos do projeto}

A seguir descreveremos de forma rápida os arquivos que estão no projeto.

\subsubsection*{cell.\+cpp}

Responsável por guardar os status de cada célula, viva ou morta, e seu número de vizinhos.

\subsubsection*{gen.\+cpp}

Guarda todas as coordenadas de elementos vivos de cada geração para ver se no futuro elas se repetem.

\subsubsection*{grid.\+cpp}

Cria a matriz e verificas, de acordo com as regras, se na próxima geração uma célula vai estar viva ou morta.

\subsubsection*{life.\+cpp}

Gerencia todo o grid chamando o método de salvar geração, verificando estábilidade e preparando o grid para a próxima geração.

\subsubsection*{life\+\_\+game.\+cpp}

Faz a leitura do arquivo de entrada pegando número de linhas, colunas, caractere desejado para indicar vida e aloca a matriz inicial. 